%%%%%%%%%%%%%%%%%%%%%%%%%%%%%%%%%%%%%%%%%
% Programming/Coding Assignment
% LaTeX Template
%
% This template has been downloaded from:
% http://www.latextemplates.com
%
% Original author:
% Ted Pavlic (http://www.tedpavlic.com)
%
% Note:
% The \lipsum[#] commands throughout this template generate dummy text
% to fill the template out. These commands should all be removed when 
% writing assignment content.
%
% This template uses a Perl script as an example snippet of code, most other
% languages are also usable. Configure them in the "CODE INCLUSION 
% CONFIGURATION" section.
%
%%%%%%%%%%%%%%%%%%%%%%%%%%%%%%%%%%%%%%%%%

%----------------------------------------------------------------------------------------
%	PACKAGES AND OTHER DOCUMENT CONFIGURATIONS
%----------------------------------------------------------------------------------------

\documentclass{article}

\usepackage{fancyhdr} % Required for custom headers
\usepackage{lastpage} % Required to determine the last page for the footer
\usepackage{extramarks} % Required for headers and footers
\usepackage[usenames,dvipsnames]{color} % Required for custom colors
\usepackage{graphicx} % Required to insert images
\usepackage{listings} % Required for insertion of code
\usepackage{courier} % Required for the courier font
\usepackage{lipsum} % Used for inserting dummy 'Lorem ipsum' text into the template

% Margins
\topmargin=-0.45in
\evensidemargin=0in
\oddsidemargin=0in
\textwidth=6.5in
\textheight=9.0in
\headsep=0.25in

\linespread{1.1} % Line spacing

% Set up the header and footer
\pagestyle{fancy}
\lhead{\hmwkAuthorName} % Top left header
\chead{\hmwkClass\ (\hmwkClassInstructor): \hmwkTitle} % Top center head
\rhead{\firstxmark} % Top right header
\lfoot{\lastxmark} % Bottom left footer
\cfoot{} % Bottom center footer
\rfoot{Page\ \thepage\ of\ \protect\pageref{LastPage}} % Bottom right footer
\renewcommand\headrulewidth{0.4pt} % Size of the header rule
\renewcommand\footrulewidth{0.4pt} % Size of the footer rule

\setlength\parindent{0pt} % Removes all indentation from paragraphs

%----------------------------------------------------------------------------------------
%	CODE INCLUSION CONFIGURATION
%----------------------------------------------------------------------------------------

\definecolor{MyDarkGreen}{rgb}{0.0,0.4,0.0} % This is the color used for comments
\lstloadlanguages{Perl} % Load Perl syntax for listings, for a list of other languages supported see: ftp://ftp.tex.ac.uk/tex-archive/macros/latex/contrib/listings/listings.pdf
\lstset{language=Matlab, % Use Perl in this example
        frame=single, % Single frame around code
        basicstyle=\tiny\ttfamily, % Use small true type font
        keywordstyle=[1]\color{Blue}\bf, % Perl functions bold and blue
        keywordstyle=[2]\color{Purple}, % Perl function arguments purple
        keywordstyle=[3]\color{Blue}\underbar, % Custom functions underlined and blue
        identifierstyle=, % Nothing special about identifiers                                         
        commentstyle=\usefont{T1}{pcr}{m}{sl}\color{MyDarkGreen}\small, % Comments small dark green courier font
        stringstyle=\color{Purple}, % Strings are purple
        showstringspaces=false, % Don't put marks in string spaces
        tabsize=5, % 5 spaces per tab
        %
        % Put standard Perl functions not included in the default language here
        morekeywords={rand},
        %
        % Put Perl function parameters here
        morekeywords=[2]{on, off, interp},
        %
        % Put user defined functions here
        morekeywords=[3]{test},
       	%
        morecomment=[l][\color{Blue}]{...}, % Line continuation (...) like blue comment
        numbers=left, % Line numbers on left
        firstnumber=1, % Line numbers start with line 1
        numberstyle=\tiny\color{Blue}, % Line numbers are blue and small
        stepnumber=1 % Line numbers go in steps of 5
}

% Creates a new command to include a perl script, the first parameter is the filename of the script (without .pl), the second parameter is the caption
\newcommand{\script}[2]{
\begin{itemize}
\item[]\lstinputlisting[caption=#2,label=#1]{#1}
\end{itemize}
}
%----------------------------------------------------------------------------------------
%	DOCUMENT STRUCTURE COMMANDS
%	Skip this unless you know what you're doing
%----------------------------------------------------------------------------------------

% Header and footer for when a page split occurs within a problem environment
\newcommand{\enterProblemHeader}[1]{
\nobreak\extramarks{#1}{#1 continued on next page\ldots}\nobreak
\nobreak\extramarks{#1 (continued)}{#1 continued on next page\ldots}\nobreak
}

% Header and footer for when a page split occurs between problem environments
\newcommand{\exitProblemHeader}[1]{
\nobreak\extramarks{#1 (continued)}{#1 continued on next page\ldots}\nobreak
\nobreak\extramarks{#1}{}\nobreak
}

\setcounter{secnumdepth}{0} % Removes default section numbers
\newcounter{homeworkProblemCounter} % Creates a counter to keep track of the number of problems

\newcommand{\homeworkProblemName}{}
\newenvironment{homeworkProblem}[1][Exercise 1.\arabic{homeworkProblemCounter}]{ % Makes a new environment called homeworkProblem which takes 1 argument (custom name) but the default is "Problem #"
\stepcounter{homeworkProblemCounter} % Increase counter for number of problems
\renewcommand{\homeworkProblemName}{#1} % Assign \homeworkProblemName the name of the problem
\section{\homeworkProblemName} % Make a section in the document with the custom problem count
\enterProblemHeader{\homeworkProblemName} % Header and footer within the environment
}{
\exitProblemHeader{\homeworkProblemName} % Header and footer after the environment
}

\newcommand{\problemAnswer}[1]{ % Defines the problem answer command with the content as the only argument
\noindent\framebox[\columnwidth][c]{\begin{minipage}{0.98\columnwidth}#1\end{minipage}} % Makes the box around the problem answer and puts the content inside
}

\newcommand{\homeworkSectionName}{}
\newenvironment{homeworkSection}[1]{ % New environment for sections within homework problems, takes 1 argument - the name of the section
\renewcommand{\homeworkSectionName}{#1} % Assign \homeworkSectionName to the name of the section from the environment argument
\subsection{\homeworkSectionName} % Make a subsection with the custom name of the subsection
\enterProblemHeader{\homeworkProblemName\ [\homeworkSectionName]} % Header and footer within the environment
}{
\enterProblemHeader{\homeworkProblemName} % Header and footer after the environment
}

%----------------------------------------------------------------------------------------
%	NAME AND CLASS SECTION
%----------------------------------------------------------------------------------------

\newcommand{\hmwkTitle}{Assignment\ \#1} % Assignment title
\newcommand{\hmwkDueDate}{Monday,\ September\ 16,\ 2014} % Due date
\newcommand{\hmwkClass}{CSCI-GA.2945-001} % Course/class
\newcommand{\hmwkClassInstructor}{Margaret Wright} % Teacher/lecturer
\newcommand{\hmwkAuthorName}{Wojciech Zaremba} % Your name

%----------------------------------------------------------------------------------------
%	TITLE PAGE
%----------------------------------------------------------------------------------------

\title{
\vspace{2in}
\textmd{\textbf{\hmwkClass:\ \hmwkTitle}}\\
\normalsize\vspace{0.1in}\small{Due\ on\ \hmwkDueDate}\\
\vspace{0.1in}\large{\textit{\hmwkClassInstructor}}
\vspace{3in}
}

\author{\textbf{\hmwkAuthorName}}
\date{} % Insert date here if you want it to appear below your name

%----------------------------------------------------------------------------------------

\begin{document}

\maketitle

%----------------------------------------------------------------------------------------
%	TABLE OF CONTENTS
%----------------------------------------------------------------------------------------

%\setcounter{tocdepth}{1} % Uncomment this line if you don't want subsections listed in the ToC

\newpage
\tableofcontents
\newpage

%----------------------------------------------------------------------------------------
%	PROBLEM 1
%----------------------------------------------------------------------------------------

% To have just one problem per page, simply put a \clearpage after each problem

\begin{homeworkProblem}

\begin{homeworkSectionName}{{\bf a})}
  Function $f(x) = x^3 - 8$ evaluates to values having a different sign at $\{-2, 4\}$, i.e. 
  $f(-2) = -16$, and $f(4) = 56$. This means that there is a zero of $f$ in interval $[-2, 4]$.
\end{homeworkSectionName}
 
\begin{homeworkSectionName}{{\bf b)}}
  \script{matlab/bisection.m}{Bisection algorithm.}
  \script{matlab/ex1_1b.m}{Program to call bisection algorithm.}
  \script{matlab/results/res_ex1_1b.txt}{Execution results.}
  Results are what I have expected. Exact value is not achieved.

\end{homeworkSectionName}


\begin{homeworkSectionName}{{\bf c)}}
  Any polynomial can be uniquely factorized to the multiplication of monomials (uniquely up to the order, and
  constant multiplicative factor) $f(x) = (x-a_1)\dots(x-a_n)$. $\{a_i\}_{i=1,\dots,n}$ are all zeros of $f$, and $f$ has no
  more zero values. This means that $f(x) = (x - 1)^7$ has only zeros at $1$.


  Another proof could be based on monotonicity of $(x - 1)^7$. We have that $\partial{x}(x-1)^7 = 7(x-1)^6 \geq 0$. 
  This means that $(x - 1)^7$ is non decreasing function. Moreover, $\lim_{x \rightarrow -\infty}(x-1)^7 = -\infty$, 
  and $\lim_{x \rightarrow \infty}(x-1)^7 = \infty$. It can cross $x=0$ only once, as it is non decreasing, and as $1$ is 
  its zero, than it is the only zero point.
\end{homeworkSectionName}
 
\begin{homeworkSectionName}{{\bf d)}}
  \script{matlab/ex1_1d.m}{Program to call bisection algorithm.}
  \script{matlab/results/res_ex1_1d.txt}{Execution results.}
  Results are what I have expected. Exact value is not achieved.

\end{homeworkSectionName}


\begin{homeworkSectionName}{{\bf e)}}
  \script{matlab/ex1_1e.m}{Program to call bisection algorithm.}
  \script{matlab/results/res_ex1_1e.txt}{Execution results.}
  Error is much smaller in compare to d). Verbose formulation might make more numerical errors.
  Results are what I have expected. Exact value is not achieved.

\end{homeworkSectionName}

\end{homeworkProblem}

%----------------------------------------------------------------------------------------
%	PROBLEM 2
%----------------------------------------------------------------------------------------

\begin{homeworkProblem}
  $f(x)'' > 0$ means that $f$ is a strictly convex function, i.e. $f(ax_0 + (1 - a)x_1) < af(x_0) + (1 - a)f(x_1) $ for $ a \in (0, 1)$.
  Or in other words, that line passing through 
   $(x_0, f(x_0))$, and $(x_1, f(x_1))$ is above graph of $f$. 

  Without lost of generality, we can assume that $x_0 < x_1$ and that $f(x_0) < 0$ and $f(x_1) > 0$. 
  $x_2$ is defined as point lying on intersection of line passing through $(x_0, f(x_0))$, and $(x_1, f(x_1))$. 
  It means that $f(x_2) < 0$. Regula falsi algorithm will keep points $x_2, x_1$. We can notice that interval $[x_2, x_1]$ 
  has the same properties as $[x_0, x_1]$, i.e. $f(x)'' > 0$, $x_0 < x_1$, $x_2 < x_1$, and $x_0 < 0$, $x_2 < 0$. 
  It means that point $x_1$ will be chosen in all future iterations of regula falsi.

\end{homeworkProblem}

%----------------------------------------------------------------------------------------
%	PROBLEM 3
%----------------------------------------------------------------------------------------

\begin{homeworkProblem}
  
\begin{homeworkSectionName}{{\bf a)}}
  In general $x_{n + 1} = x_n - \frac{f(x_n)}{f(x_n)'}$. We have that $f(x) = x^3 - c$, and $f(x)' = 3x^2$. This gives us an
  update rule $x_{n + 1} = x_n -  \frac{x_n^3 - c}{3x_n^2} = \frac{2}{3}x_n - \frac{c}{3x_n^2}$
\end{homeworkSectionName}

\begin{homeworkSectionName}{{\bf b)}}
  \script{matlab/newton.m}{Newton algorithm.}
  \script{matlab/secant.m}{Secant method.}
  \script{matlab/regula_falsi.m}{Regula falsi.}
  \script{matlab/wheeler.m}{Wheeler method.}
\end{homeworkSectionName}

\begin{homeworkSectionName}{{\bf c)}}
  \script{matlab/ex1_3c.m}{Implementation}
  \script{matlab/results/res_ex1_3c_i.txt}{Results for i)}
  \script{matlab/results/res_ex1_3c_ii.txt}{Results for ii)}
  \script{matlab/results/res_ex1_3c_iii.txt}{Results for iii)}
  \script{matlab/results/res_ex1_3c_iv.txt}{Results for iv)}

  Newton method converges very fast (quadratic speed of convergence). 
  Close to convergence point precision doubles.
\end{homeworkSectionName}

\begin{homeworkSectionName}{{\bf d)}}
  \script{matlab/ex1_3d.m}{Implementation}
  \script{matlab/results/res_ex1_3d_i.txt}{Results for i)}
  \script{matlab/results/res_ex1_3d_ii.txt}{Results for ii)}
  \script{matlab/results/res_ex1_3d_iii.txt}{Results for iii)}
  \script{matlab/results/res_ex1_3d_iv.txt}{Results for iv)}

  Secant method converges very fast. 
\end{homeworkSectionName}


\begin{homeworkSectionName}{{\bf e)}}
  \script{matlab/ex1_3e.m}{Implementation}
  \script{matlab/results/res_ex1_3e_i.txt}{Results for i)}
  \script{matlab/results/res_ex1_3e_ii.txt}{Results for ii)}
  \script{matlab/results/res_ex1_3e_iii.txt}{Results for iii)}
  \script{matlab/results/res_ex1_3e_iv.txt}{Results for iv)}

  Sometimes it takes a very long time for regula falsi to converge.
\end{homeworkSectionName}

\end{homeworkProblem}

%----------------------------------------------------------------------------------------

\end{document}
